\documentclass[12pt,a4paper]{article}
\usepackage{natbib}         % Pour la bibliographie
\usepackage{url}            % Pour citer les adresses web
\usepackage[T1]{fontenc}    % Encodage des accents
\usepackage[utf8]{inputenc}
\usepackage[english]{babel}
\usepackage{numprint}       % Histoire que les chiffres soient bien

\usepackage{amsmath}        % La base pour les maths
\usepackage{mathrsfs}       % Quelques symboles suppl�mentaires
\usepackage{amssymb}        % encore des symboles.
\usepackage{amsfonts}       % Des fontes, eg pour \mathbb.

\usepackage[svgnames]{xcolor} % De la couleur
\usepackage{geometry}       % G�rer correctement la taille

%%% Si jamais vous voulez changer de police: d�commentez les trois 
%\usepackage{tgpagella}
%\usepackage{tgadventor}
%\usepackage{inconsolata}

%%% Pour L'utilisation de Python
\usepackage{pythontex}

\usepackage{graphicx} % inclusion des graphiques
\usepackage{wrapfig}  % Dessins dans le texte.

\usepackage{tikz}     % Un package pour les dessins (utilis� pour l'environnement {code})
\usepackage[framemethod=TikZ]{mdframed}

\usepackage{enumitem}

% Un environnement pour bien pr�senter le code informatique
\newenvironment{code}{%
\begin{mdframed}[linecolor=Green,innerrightmargin=30pt,innerleftmargin=30pt,
backgroundcolor=Black!5,
skipabove=10pt,skipbelow=10pt,roundcorner=5pt,
splitbottomskip=6pt,splittopskip=12pt]
}{%
\end{mdframed}
}

% Mettez votre titre et votre nom ci-apr�s
\title{Course of Web Information Retrieval \\
Homework 3}
\author{Simone Agostinelli 1523559, Giacomo Vettraino 1594722}
%% � d�commenter si vous ne voulez pas que la date apparaisse explicitement
\date{}

% Un raccourci pour composer les unit�s correctement (en droit)
% Exemple: $v = 10\U{m.s^{-1}}$
\newcommand{\U}[1]{~\mathrm{#1}}

% Pour discuter avec le prof dans le document: le premier argument est 
% le nom de celui qui fait la remarque et le second la remarque 
% proprement dite: \question{jj}{Que voulez-vous dire par l� ?}
% \reponse{Droopy}{I'm very happy...}
\usepackage{todonotes}
\newcommand{\question}[2]{\todo[inline,author=#1]{#2}}
\newcommand{\reponse}[2]{\todo[inline,color=green,author=#1]{#2}}

% Les guillemets \ofg{par exemple}
\newcommand{\ofg}[1]{\og{}#1\fg{}}
% Le d des d�riv�es doit �tre droit: \frac{\dd x}{\dd t}
\newcommand{\dd}{\text{d}}


% NB: le script TeXcount permet de compter les mots utilis�s dans chaque section d'un document LaTeX. Vous en trouverez une version en ligne � l'adresse
% http://app.uio.no/ifi/texcount/online.php
% Il suffit d'y copier l'ensemble du pr�sent document (via Ctrl-A/Ctrl-C puis Ctrl-V dans la fen�tre idoine) pour obtenir le r�capitulatif tout en bas de la page qui s'ouvre alors.

% Pour r�cup�rer les bonnes entr�es bibliographiques, je vous conseille l'usage de scholar.google.fr pour les recherches
% et la r�cup�ration des entr�e BibTeX comme d�crit dans cette vid�o: https://www.youtube.com/watch?v=X-9T2Oaj-5A

% Quelques macros utiles
% La d�riv�e temporelle, tellement courante en physique, avec les d droits
\newcommand{\ddt}[1]{\frac{\dd #1}{\dd t}}
% Des parenth�ses, crochets et accolades qui s'adaptent automatiquement � la taille de ce qu'il y a dedans
\newcommand{\pa}[1]{\left(#1\right)}
\newcommand{\pac}[1]{\left[#1\right]}
\newcommand{\paa}[1]{\left\{#1\right\}}
% Un raccourci pour �crire une constante
\newcommand{\cte}{\text{C}^{\text{te}}}
% Pour faire des indices en mode texte (comme les �nergie potentielles)
\newcommand{\e}[1]{_{\text{#1}}}
% Le produit vectoriel a un nom bizarre:
\newcommand{\vectoriel}{\wedge}

\usepackage{multirow}

\usepackage{color}
\usepackage{listings}
\usepackage{setspace}
\definecolor{Code}{rgb}{0,0,0}
\definecolor{Decorators}{rgb}{0.5,0.5,0.5}
\definecolor{Numbers}{rgb}{0.5,0,0}
\definecolor{MatchingBrackets}{rgb}{0.25,0.5,0.5}
\definecolor{Keywords}{rgb}{0,0,1}
\definecolor{self}{rgb}{0,0,0}
\definecolor{Strings}{rgb}{0,0.63,0}
\definecolor{Comments}{rgb}{0,0.63,1}
\definecolor{Backquotes}{rgb}{0,0,0}
\definecolor{Classname}{rgb}{0,0,0}
\definecolor{FunctionName}{rgb}{0,0,0}
\definecolor{Operators}{rgb}{0,0,0}
\definecolor{Background}{rgb}{0.98,0.98,0.98}
\lstdefinelanguage{Python}{
numbers=left,
numberstyle=\footnotesize,
numbersep=1em,
xleftmargin=1em,
framextopmargin=2em,
framexbottommargin=2em,
showspaces=false,
showtabs=false,
showstringspaces=false,
frame=l,
tabsize=4,
% Basic
basicstyle=\ttfamily\small\setstretch{1},
backgroundcolor=\color{Background},
% Comments
commentstyle=\color{Comments}\slshape,
% Strings
stringstyle=\color{Strings},
morecomment=[s][\color{Strings}]{"""}{"""},
morecomment=[s][\color{Strings}]{'''}{'''},
% keywords
morekeywords={import,from,class,def,for,while,if,is,in,elif,else,not,and,or,print,break,continue,return,True,False,None,access,as,,del,except,exec,finally,global,import,lambda,pass,print,raise,try,assert},
keywordstyle={\color{Keywords}\bfseries},
% additional keywords
morekeywords={[2]@invariant,pylab,numpy,np,scipy},
keywordstyle={[2]\color{Decorators}\slshape},
emph={self},
emphstyle={\color{self}\slshape},
%
}

 
\usepackage{minted}

\begin{document}

\maketitle

\section{Ham/Spam Classifier}

\subsection{KNN Classifier}



\subsubsection{Dataset}
Our dataset is composed by 347 text files representing ham and spam comments related to YouTube videos. These files are composed according to the following table:\\

\begin{center}
\begin{tabular}{ |l|c|c| }
\hline
 & Training Set & Test Set \\ \hline
\multirow{3}{*}{Spam}
 & & \\
 & 122 &  53\\
  & & \\

 \hline
\multirow{3}{*}{Ham}
 & & \\
 & 120 &  52\\
  & & \\
 \hline
\end{tabular}
\end{center}

\subsubsection{Vectorizer and Classifier Parameters}
Firstly we consider the process of vectorization that we have to apply in order to transform our text files into a matrix of TfIdf features. For this step we have:\\
\begin{itemize}[label={}]
	\item $ \textbf{tokenizer} \in \{None, stemming\_tokenizer, stemming\_tokenizer\_stopwords\_filter\}$\\   override the string tokenization function if its value is different from $None$;
    	\item  $ \textbf{ngram\_range} \in \{ (1, 1), (1, 2),(1,3)\}$\\ specify the lower and upper boundary of the range of n-values for different n-grams to be extracted;\\
\end{itemize}

 While, for the classification phase we have:\\
\begin{itemize}[label={}]
	\item $ \textbf{n\_neighbors} \in \{1,3,5,7,9\}$\\ specify the number of neighbors to use for classifying points;
    	\item  $ \textbf{weigths} \in \{ "uniform", "distance"\}$ \\specify the weight function in prediction ;
\end{itemize}


\subsubsection{Training-Validation Phase}
Training and validation are performed by using the sklearn Python function \textbf{GridSearchCV} in an automated fashion. Indeed, it performs an exhaustive search of the best parameters values configuration by fitting the specific model (or a series of models transformation if pipeline is used); furthermore, it can output the best estimator and the best parameters configuration by computing scores on this combinations using a scoring function chosen by the user.\\
In particular, in our case, firstly we decided to use \textbf{Pipeline} sklearn function, in order to construct the different steps that we want to cross-validate: vectorization and classifier fitting.
\textbf{Pipeline} function has input parameter:\\
\begin{itemize}[label={}]
	\item $ \textbf{steps} $ \\ List of (name, transform) tuples  that are chained, in the order in which they are chained, with the last object an estimator.\\
\end{itemize}
 
\begin{lstlisting}[language=Python]
vectorizer = TfidfVectorizer(strip_accents= None,
preprocessor = None,)
knn = KNeighborsClassifier()
pipeline = Pipeline([('vect', vectorizer), ('knn', knn),])
\end{lstlisting}

Then, we can directly make use of \textbf{GridSearchCV} function, in order to setup the automated search for the best parameter configuration.\\
\textbf{GridSearchCV} has as input parameters:\\
\begin{itemize}[label={}]
	\item $ \textbf{estimator} $ \\ This is the estimator on which we want to perform our grid search; in this case we set it to the pipeline just created, as its last step represent an estimator, our KNN Classifier;\\
	\item $ \textbf{param\_grid} $ \\ Dictionary with parameters names (string) as keys and lists of parameter settings to try as values, or a list of such dictionaries, in which case the grids spanned by each dictionary in the list are explored; this enables searching over any sequence of parameter settings;\\
	\item $ \textbf{scoring} $ \\A string (see model evaluation documentation) or a scorer callable object / function with signature scorer(estimator, X, y); in this case Matthews correlation coefficient is used;\\
	\item $ \textbf{n\_jobs} $ \\ Number of jobs to run in parallel, in this way we can control the parallelism of our program, in order to speed up it running time;\\
	\item $ \textbf{cv} $ \\determines the cross-validation splitting strategy; an integer specify the number of folds in a  KFold, 10 in our case;\\
\end{itemize}

\begin{lstlisting}[language=Python]
parameters = {
'vect__tokenizer': [None, stemming_tokenizer,
stemming_tokenizer_stopwords_filter],
'vect__ngram_range': [(1, 1), (1, 2),(1,3)],
'knn__n_neighbors': [1,3,5,7,9],
'knn__weights': ["uniform", "distance"]
}
grid_search = GridSearchCV(
pipeline,
parameters,
scoring = metrics.make_scorer(metrics.matthews_corrcoef),
cv = 10,
n_jobs = 4)
\end{lstlisting}

\subsubsection{Best Parameters Values}
After performing cross-validation, as explained before, we have found that the best parameters configuration is:
\begin{table}[!h]
\begin{center}
\begin{tabular}{| c | c |}
\hline
\textbf{Parameter} & \textbf{Value} \\ \hline
knn\_\_n\_neighbors & 9\\ \hline
knn\_\_n\_weights & distance\\ \hline
vect\_\_ngram\_range & (1,2)\\ \hline
vect\_\_tokenizer & None\\ \hline

\end{tabular}
\end{center}
\end{table}

\subsubsection{Results for Classification}
Output of \textbf{metrics.classification\_report}:\\
\begin{lstlisting}[language=Python]
----------------------------------------------------
             precision    recall  f1-score   support

        Ham       0.83      0.96      0.89        52
       Spam       0.96      0.81      0.88        53

avg / total       0.90      0.89      0.89       105
----------------------------------------------------
\end{lstlisting}
The \textbf{Confusion Matrix}:\\
\begin{center}
\begin{tabular}{ |l|c|c| }
\hline
 & Predicted-Ham & Predicted-Spam \\ \hline
\multirow{3}{*}{True-Ham}
 & & \\
 & 50 &  2\\
  & & \\

 \hline
\multirow{3}{*}{True-Spam}
 & & \\
 & 10 &  43\\
  & & \\
 \hline
\end{tabular}
\end{center}
The \textbf{Normalized-Accuracy} value: 0.885714285714\\
The \textbf{Mattiews Correlation Coefficient}: 0.780832919631\\

\section{Sentiment Analysis}


\subsection{KNN Classifier}



\subsubsection{Dataset}
Our dataset is composed by  1115 text files representing positive and negative sentences. These files are composed according to the following table:\\

\begin{center}
\begin{tabular}{ |l|c|c| }
\hline
 & Training Set & Test Set \\ \hline
\multirow{3}{*}{Negative}
 & & \\
 & 249 &  250\\
  & & \\

 \hline
\multirow{3}{*}{Positive}
 & & \\
 & 308 &  308\\
  & & \\
 \hline
\end{tabular}
\end{center}
\subsubsection{Vectorizer and Classifier Parameters}
Firstly we consider the process of vectorization that we have to apply in order to transform our text files into a matrix of TfIdf features. For this step we have:\\
\begin{itemize}[label={}]
	\item $ \textbf{tokenizer} \in \{None, stemming\_tokenizer, stemming\_tokenizer\_stopwords\_filter\}$ \\  override the string tokenization function if its value is different from $None$;
    	\item  $ \textbf{ngram\_range} \in \{ (1, 1), (1, 2),(1,3)\}$\\ specify the lower and upper boundary of the range of n-values for different n-grams to be extracted;\\
\end{itemize}

 While, for the classification phase we have:\\
\begin{itemize}[label={}]
	\item $ \textbf{n\_neighbors} \in \{1,3,5,7,9\}$ \\ specify the number of neighbors to use for classifying points;
    	\item  $ \textbf{weigths} \in \{ uniform, distance\}$\\ specify the weight function in prediction ;\\
\end{itemize}


\subsubsection{Training-Validation Phase}
\subsubsection{Best Parameters Values}
After performing cross-validation, as explained before, we have found that the best parameters configuration is:
\begin{table}[!h]
\begin{center}
\begin{tabular}{| c | c |}
\hline
\textbf{Parameter} & \textbf{Value} \\ \hline
knn\_\_n\_neighbors & 3\\ \hline
knn\_\_n\_weights & distance\\ \hline
vect\_\_ngram\_range & (1,2)\\ \hline
vect\_\_tokenizer & stemming\_tokenizer\_stopwords\_filter\\ \hline

\end{tabular}
\end{center}
\end{table}

\subsubsection{Results for Classification}
Output of \textbf{metrics.classification\_report}:\\
\begin{lstlisting}[language=Python]
----------------------------------------------------
             precision    recall  f1-score   support

   Positive       0.85      0.94      0.89       308
   negative       0.92      0.80      0.85       250

avg / total       0.88      0.88      0.88       558
----------------------------------------------------
\end{lstlisting}

The \textbf{Confusion Matrix}:\\
\begin{center}
\begin{tabular}{ |l|c|c| }
\hline
 & Predicted-Ham & Predicted-Spam \\ \hline
\multirow{3}{*}{True-Ham}
 & & \\
 & 290 & 1 8\\
  & & \\

 \hline
\multirow{3}{*}{True-Spam}
 & & \\
 & 51 &  199\\
  & & \\
 \hline
\end{tabular}
\end{center}
The \textbf{Normalized-Accuracy} value: 0.876344086022\\
The \textbf{Mattiews Correlation Coefficient}: 0.752375671874\\


\subsection{MultinomialNB Classifier}


\subsubsection{Dataset}
Our dataset is composed by  1115 text files representing positive and negative sentences. These files are composed according to the following table:\\

\begin{center}
\begin{tabular}{ |l|c|c| }
\hline
 & Training Set & Test Set \\ \hline
\multirow{3}{*}{Negative}
 & & \\
 & 249 &  250\\
  & & \\

 \hline
\multirow{3}{*}{Positive}
 & & \\
 & 308 &  308\\
  & & \\
 \hline
\end{tabular}
\end{center}

\subsubsection{Vectorizer and Classifier Parameters}
Firstly we consider the process of vectorization that we have to apply in order to transform our text files into a matrix of TfIdf features. For this step we have:\\
\begin{itemize}[label={}]
	\item $ \textbf{tokenizer} \in \{None, stemming\_tokenizer, stemming\_tokenizer\_stopwords\_filter\}$\\   override the string tokenization function if its value is different from $None$;
    	\item  $ \textbf{ngram\_range} \in \{ (1, 1), (1, 2),(1,3)\}$\\ specify the lower and upper boundary of the range of n-values for different n-grams to be extracted;\\
\end{itemize}

 While, for the classification phase we have:\\
\begin{itemize}[label={}]
	\item $ \textbf{alpha} \in \{0.001, 0.01,  1, 10\}$ specify the additive smoothing parameter;\\
\end{itemize}

\subsubsection{Parallelism}


\subsubsection{Best Parameters Values}
After performing cross-validation, as explained before, we have found that the best parameters configuration is:
\begin{table}[!h]
\begin{center}
\begin{tabular}{| c | c |}
\hline
\textbf{Parameter} & \textbf{Value} \\ \hline
mnb\_\_alpha & 1\\ \hline
vect\_\_ngram\_range & (1,1)\\ \hline
vect\_\_tokenizer & stemming\_tokenizer\_stopwords\_filter\\ \hline
\end{tabular}
\end{center}
\end{table}

\subsubsection{Results for Classification}
Output of \textbf{metrics.classification\_report}:\\
\begin{lstlisting}[language=Python]
----------------------------------------------------
             precision    recall  f1-score   support

   Positive       0.90      0.98      0.94       308
   negative       0.98      0.86      0.92       250

avg / total       0.93      0.93      0.93       558
----------------------------------------------------
\end{lstlisting}

The \textbf{Confusion Matrix}:\\
\begin{center}
\begin{tabular}{ |l|c|c| }
\hline
 & Predicted-Ham & Predicted-Spam \\ \hline
\multirow{3}{*}{True-Ham}
 & & \\
 & 303 &  5\\
  & & \\

 \hline
\multirow{3}{*}{True-Spam}
 & & \\
 & 34 &  216\\
  & & \\
 \hline
\end{tabular}
\end{center}
The \textbf{Normalized-Accuracy} value: 0.930107526882\\
The \textbf{Mattiews Correlation Coefficient}: 0.862006201146\\


\subsection{SVC}


\subsubsection{Dataset}
Our dataset is composed by  1115 text files representing positive and negative sentences. These files are composed according to the following table:\\

\begin{center}
\begin{tabular}{ |l|c|c| }
\hline
 & Training Set & Test Set \\ \hline
\multirow{3}{*}{Negative}
 & & \\
 & 249 &  250\\
  & & \\

 \hline
\multirow{3}{*}{Positive}
 & & \\
 & 308 &  308\\
  & & \\
 \hline
\end{tabular}
\end{center}

\subsubsection{Vectorizer and Classifier Parameters}
Firstly we consider the process of vectorization that we have to apply in order to transform our text files into a matrix of TfIdf features. For this step we have:\\
\begin{itemize}[label={}]
	\item $ \textbf{tokenizer} \in \{None, stemming\_tokenizer, stemming\_tokenizer\_stopwords\_filter\}$ \\  override the string tokenization function if its value is different from $None$;
    	\item  $ \textbf{ngram\_range} \in \{ (1, 1), (1, 2),(1,3)\}$\\ specify the lower and upper boundary of the range of n-values for different n-grams to be extracted;\\
\end{itemize}

 While, for the classification phase we have:\\
\begin{itemize}[label={}]
	\item $ \textbf{C} \in \{0.01, 0.1, 1.0, 10.0, 100.0\}$ \\ specify the penalty parameter C value of the error term ;\\
\end{itemize}

\subsubsection{Parallelism}


\subsubsection{Best Parameters Values}
After performing cross-validation, as explained before, we have found that the best parameters configuration is:
\begin{table}[!h]
\begin{center}
\begin{tabular}{| c | c |}
\hline
\textbf{Parameter} & \textbf{Value} \\ \hline
svc\_\_C : 10\\ \hline
vect\_\_ngram\_range & (1,3)\\ \hline
vect\_\_tokenizer & stemming\_tokenizer\_stopwords\_filter\\ \hline
\end{tabular}
\end{center}
\end{table}

\subsubsection{Results for Classification}
Output of \textbf{metrics.classification\_report}:\\
\begin{lstlisting}[language=Python]
----------------------------------------------------
             precision    recall  f1-score   support

   Positive       0.96      0.97      0.97       308
   negative       0.97      0.95      0.96       250

avg / total       0.96      0.96      0.96       558
----------------------------------------------------
\end{lstlisting}

The \textbf{Confusion Matrix}:\\
\begin{center}
\begin{tabular}{ |l|c|c| }
\hline
 & Predicted-Ham & Predicted-Spam \\ \hline
\multirow{3}{*}{True-Ham}
 & & \\
 & 300 &  8\\
  & & \\

 \hline
\multirow{3}{*}{True-Spam}
 & & \\
 & 13 &  237\\
  & & \\
 \hline
\end{tabular}
\end{center}
The \textbf{Normalized-Accuracy} value: 0.962365591398\\
The \textbf{Mattiews Correlation Coefficient}: 0.923917742518\\

\end{document}